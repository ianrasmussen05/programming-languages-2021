\documentclass{article}

\usepackage{amsthm}
\usepackage{amsfonts}
\usepackage{amsmath}
\usepackage{amssymb}
\usepackage{fullpage}
\usepackage[usenames]{color}
\usepackage{hyperref}
  \hypersetup{
    colorlinks = true,
    urlcolor = blue,       % color of external links using \href
    linkcolor= blue,       % color of internal links 
    citecolor= blue,       % color of links to bibliography
    filecolor= blue,        % color of file links
    }
    
\usepackage{listings}

\definecolor{dkgreen}{rgb}{0,0.6,0}
\definecolor{gray}{rgb}{0.5,0.5,0.5}
\definecolor{mauve}{rgb}{0.58,0,0.82}

\lstset{frame=tb,
  language=haskell,
  aboveskip=3mm,
  belowskip=3mm,
  showstringspaces=false,
  columns=flexible,
  basicstyle={\small\ttfamily},
  numbers=none,
  numberstyle=\tiny\color{gray},
  keywordstyle=\color{blue},
  commentstyle=\color{dkgreen},
  stringstyle=\color{mauve},
  breaklines=true,
  breakatwhitespace=true,
  tabsize=3
}


\title{CPSC-354 Report}
\author{Your Name  \\ Chapman University}

\date{\today}

\begin{document}

\maketitle

\begin{abstract}
Short introduction to your report \ldots 
\end{abstract}

\tableofcontents

\section{Introduction}\label{intro}

Replace Section~\ref{intro} with your own short introduction. 

\subsection{General Remarks}

First you need to \href{https://www.latex-project.org/get/}{download and install} LaTeX.\footnote{Links are typeset in blue, but you can change the layout and color of the links if you locate the  \texttt{hypersetup} command.}
%
Alternatively, you can use an online editor such as \href{https://www.overleaf.com/learn}{Overleaf}. I prefer to have my own installation, but to get started Overleaf may be easier. 

 
\medskip\noindent
LaTeX is a markup language (as is, for example, HTML). The source code is in a \verb+.tex+ file and needs to be compiled for viewing, usually to \verb+.pdf+.


\medskip\noindent
If you want to change the default layout, you need to type commands. For example, \verb+\medskip+ inserts a medium vertical space and \verb+\noindent+ starts a paragraph without indentation.
 
\medskip\noindent
Mathematics is typeset between double dollars, for example $$x+y=y+x.$$


\subsection{LaTeX Resources}

I start a new subsection, so that you can see how it appears in the table of contents.

\begin{itemize}
\item This is how you itemize in LaTeX.
\item I think a good way to learn LaTeX is by starting from this template file and build it up step by step. Often stackoverflow will answer your questions. But here are a few resources:
  \begin{enumerate}
  \item \href{https://www.overleaf.com/learn/latex/Learn_LaTeX_in_30_minutes}{Learn LaTeX in 30 minutes}
  \item \href{https://www.latex-project.org/}{LaTeX – A document preparation system}\end{enumerate}
\end{itemize}

\subsection{Plagiarism}

To avoid plagiarism, make sure that in addition to \cite{PL} you also cite all the external sources you use.

\section{Haskell}\label{haskell}

This section will contain your own introduction to Haskell. 

\medskip\noindent
To typeset Haskell there are several possibilities. For the example below I took the LaTeX code from \href{https://stackoverflow.com/a/3175141/4600290}{stackoverflow} and the Haskell code from \href{https://hackmd.io/@alexhkurz/HylLKujCP}{my tutorial}.

\begin{lstlisting}
-- run the transition function on a word and a state
run :: (State -> Char -> State) -> State -> [Char] -> State
run delta q [] = q
run delta q (c:cs) = run delta (delta q c) cs 
\end{lstlisting}

\medskip\noindent
This works well for short snippets of code. For entire programs, it is better to have external links to, for example, Github or \href{https://replit.com/@alexhkurz/automata01#main.hs}{Replit} (click on the "Run" button and/or the "Code" tab).


\section{Programming Languages Theory}

In this section you will show what you learned about the theory of programming languages. 

\section{Project}

In this section you will describe a short project. It can either be in Haskell or of a theoretical nature,

\section{Conclusions}\label{conclusions}
Short conclusion. 

\begin{thebibliography}{99}
\bibitem[PL]{PL} \href{https://github.com/alexhkurz/programming-languages-2021/blob/main/README.md}{Programming Languages 2021}, Chapman University, 2021.
\end{thebibliography}

\end{document}
